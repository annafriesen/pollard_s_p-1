% Type of the document
\documentclass{beamer}

% elementary packages:
\usepackage{graphicx}
\usepackage[latin1]{inputenc}
\usepackage[T1]{fontenc}
\usepackage[english]{babel}
\usepackage{listings}
\usepackage{xcolor}
\usepackage{eso-pic}
\usepackage{mathrsfs}
\usepackage{url}
\usepackage{amssymb}
\usepackage{amsmath}
\usepackage{multirow}
\usepackage{hyperref}
\usepackage{booktabs}

% additional packages
\usepackage{bbm}

% packages supplied with ise-beamer:
\usepackage{cooltooltips}
\usepackage{colordef}
\usepackage{beamerdefs}
\usepackage{lvblisting}

% Change the pictures here:
% logobig and logosmall are the internal names for the pictures: do not modify them. 
% Pictures must be supplied as JPEG, PNG or, to be preferred, PDF
\pgfdeclareimage[height=2cm]{logobig}{hulogo}
% Supply the correct logo for your class and change the file name to "logo". The logo will appear in the lower
% right corner:
\pgfdeclareimage[height=0.7cm]{logosmall}{Figures/LOB_Logo}

% Title page outline:
% use this number to modify the scaling of the headline on title page
\renewcommand{\titlescale}{1.0}
% the title page has two columns, the following two values determine the percentage each one should get
\renewcommand{\titlescale}{1.0}
\renewcommand{\leftcol}{0.6}

% Define the title.Don't forget to insert an abbreviation instead 
% of "title for footer". It will appear in the lower left corner:
\title[Prime Factorization]{Prime Factorization}
% Define the authors:
\authora{Anna Friesen} % a-c

% Define any internet addresses, if you want to display them on the title page:
\def\linka{http://lvb.wiwi.hu-berlin.de}
\def\linkb{}
\def\linkc{}
% Define the institute:
\institute{Ladislaus von Bortkiewicz Chair of Statistics \\
Humboldt--Universit�t zu Berlin \\}

% Comment the following command, if you don't want, that the pdf file starts in full screen mode:
\hypersetup{pdfpagemode=FullScreen}

%Start of the document
\begin{document}

% create the title slide, layout controlled in beamerdefs.sty and the foregoing specifications
\frame[plain]{
\titlepage
}

% The titles of the different sections of you talk, can be included via the \section command. The title will be displayed in the upper left corner. To indicate a new section, repeat the \section command with, of course, another section title
%%%%%%%%%%%%%%%%%%%%%%%%%%%%%%%%%%%%%%%%%%%%%%%%%%%%%%%%%%%%%%%%%%%%%%%%%%%%%%%%%%%%%%%%%%%%%%%%%%%%%%%%%%%%%%%%%%%%%%%%
\section{Introduction}
%%%%%%%%%%%%%%%%%%%%%%%%%%%%%%%%%%%%%%%%%%%%%%%%%%%%%%%%%%%%%%%%%%%%%%%%%%%%%%%%%%%%%%%%%%%%%%%%%%%%%%%%%%%%%%%%%%%%%%%%

% (A numbering of the slides can be useful for corrections, especially if you are
% dealing with large tex-files)
%%%%%%%%%%%%%%%%%%%%%%%%%%%%%%%%%%%%%%%%%%%%%%%%%%%%%%%%%%%%%%%%%%%%%%%%%%%%%%%%%%%%%%%%%%%%%%%%%%%%%%%%%%%%%%%%%%%%%%%%
\frame{
\frametitle{Motivation}
\begin{itemize}
\item RSA cryptosystem
\item published in 1977 at the MIT 
\item by Ron \textbf{R}ivest, Adi \textbf{S}hamir and Leonard \textbf{A}delman
\item algorithm is based on two pairs of keys
\begin{itemize}
\item first pair $(N,E)$ to encrypt data
\item second pair $(N,D)$ to decrypt the data 
\end{itemize}
\item $N$ is product of two large primes $p,q$ \\($\log(i) > 232, i = p,q$) 
\item $N$ cannot be factored by now (computationally intractable  for classical (non-quantum) computers)
\item using quantum computers, Shor's algorithm could factor $N$
\item RSA ist still used to encrypt (still secure)
\end{itemize}
}

%%%%%%%%%%%%%%%%%%%%%%%%%%%%%%%%%%%%%%%%%%%%%%%%%%%%%%%%%%%%%%%%%%%%%%%%%%%%%%%%%%%%%%%%%%%%%%%%%%%%%%%%%%%%%%%%%%%%%%%%
\frame{
%\frametitle{Motivation}

\textbf{The RSA algorithm:}
%
\begin{itemize}
\item let $OI$ be the original integer to be encrypted
\item or transfer your text into an integer (e.\,g. number in alphabet)
\end{itemize}
\begin{enumerate}
\item choose two random primes $p,q$
\begin{itemize}
\item $ p \cdot q = N > OI $ and $ p \neq q $ and $ \mid p - q \mid $ not too small
\end{itemize}
\item choose $E$ ($E$ prime suffices all conditions)
\begin{itemize}
\item $ E \in \{n \in \mathbb{N} \mid 2 \nmid n \}$ and $ E \nmid N $ and $  E > (p-1) \cdot (q-1)$
\end{itemize}
\item choose $D$
\begin{itemize}
\item $ (E \cdot D) mod ((p-1) \cdot (q-1)) = 1$
\end{itemize}
\end{enumerate}
%
\textbf{How to use the pairs of keys:}
\begin{itemize}
\item $ OI \overrightarrow{\tiny{\text{encrypt}}} CI: CI = OI^E mod (N) $ \\[0.2cm] 
\item $ CI \overrightarrow{\tiny{\text{decrypt}}} OI: OI = CI^D mod (N) $ \\[0.2cm] 
\end{itemize}
}

%%%%%%%%%%%%%%%%%%%%%%%%%%%%%%%%%%%%%%%%%%%%%%%%%%%%%%%%%%%%%%%%%%%%%%%%%%%%%%%%%%%%%%%%%%%%%%%%%%%%%%%%%%%%%%%%%%%%%%%%
\section{Prime Factorization}
%%%%%%%%%%%%%%%%%%%%%%%%%%%%%%%%%%%%%%%%%%%%%%%%%%%%%%%%%%%%%%%%%%%%%%%%%%%%%%%%%%%%%%%%%%%%%%%%%%%%%%%%%%%%%%%%%%%%%%%%

% Subsections are not visible on the actual slide, but are displayed as bookmarks in the pdf file. Their application facilitates an easy navigation trough large pdf files.
%%%%%%%%%%%%%%%%%%%%%%%%%%%%%%%%%%%%%%%%%%%%%%%%%%%%%%%%%%%%%%%%%%%%%%%%%%%%%%%%%%%%%%%%%%%%%%%%%%%%%%%%%%%%%%%%%%%%%%%%
\subsection{Trial Division}
%%%%%%%%%%%%%%%%%%%%%%%%%%%%%%%%%%%%%%%%%%%%%%%%%%%%%%%%%%%%%%%%%%%%%%%%%%%%%%%%%%%%%%%%%%%%%%%%%%%%%%%%%%%%%%%%%%%%%%%%

%%%%%%%%%%%%%%%%%%%%%%%%%%%%%%%%%%%%%%%%%%%%%%%%%%%%%%%%%%%%%%%%%%%%%%%%%%%%%%%%%%%%%%%%%%%%%%%%%%%%%%%%%%%%%%%%%%%%%%%%
\frame{
\frametitle{Trial Division}

\begin{itemize}
\item the simplest (and most simple-minded) prime factorization algorithm
\item assume $n$ is not prime, test for $2 \leq i \leq \lfloor \sqrt{n} \rfloor$ whether $i \mid n$
\item  works quite well, because most coposite numbers have small prime factors
\item inefficient if $n$ has large prime factors
\begin{itemize}
\item \textbf{efficient} algorithm has \textbf{polynomial} running time $f(\log_2(n))$, i.e. $f(\log_2(n))\;\in\;O((\log_2(n))^k) \text{ mit } k\ge 0$
\end{itemize}
\item modifications:
\begin{itemize}
\item choose $i \in \{k \in \mathbb{N} \mid 2\nmid k \}$
\item choose fixed bound B: find $i \leq B$ (often $B = 10^6$)
\item test primes $i \leq \lfloor \sqrt{r} \rfloor$ (sieve of Eratosthenes)
\end{itemize}
\end{itemize}
}

\frame{
\frametitle{Sieve of Eratosthenes}
%
\begin{itemize}
\item named by the greek mathematician Eratosthenes of Cyrene (3rd century BC)
\item invented the name \textit{sieve} for known algorithm
\item deterministic algorithm
\item creates list of primes that are smaller than or eqal to a given integer $r \ge 2$
\item running time $\log_2(r-1)\cdot (r-1)$ is \textit{exponentional}
\end{itemize}
%
}

\subsection{Modern factorization algorithms}
\frame{
\frametitle{Modern factorization algorithms}
%
\begin{itemize}
	\item devided into two groups
	\begin{itemize}
		\item special purpose algorithms
		\begin{itemize}
			\item efficiency depends on factors of number being factored
			\item e.g. Pollard (p-1) factorization method
			\item not dangerous to RSA (RSA uses large primes)
		\end{itemize}
		\item general purpose algorithms
		\begin{itemize}
			\item efficiency depends only on number being factored
			\item e.g. general number field sieve
		\end{itemize}
	\end{itemize}
\end{itemize}
%
}
%
%%%%%%%%%%%%%%%%%%%%%%%%%%%%%%%%%%%%%%%%%%%%%%%%%%%%%%%%%%%%%%%%%%%%%%
%%%%%%%%%%%%%%%%%%%%%%%%%%%%%%%%%%%%%%%%%%%%%%%%%%%%%%%%%%%%%%%%%%%%%%



\subsection{Pollard (p-1) factorization method}
%
\frame{
\frametitle{Pollard (p-1) factorization method}
%
\begin{itemize}
\item based on Fermat�s little theorem
\end{itemize}
%
\begin{theorem}[Fermat�s little theorem]
%
Let $p$ be prime ($p \in \mathbb{P}$) and $a \in \mathbb{Z}$ with $a<p$. Then it holds:
\[
p \in \mathbb{P} \;\Rightarrow\; a^{p} \; \equiv \; a \:(\text{mod}\, p)
\]
If additionally $a$ and $p$ have no common divisor, i.e. $\gcd(p,a)=1$, then it even holds:\par
%
\[
p \in \mathbb{P} \;\Rightarrow\; a^{p-1} \; \equiv \; 1 \:(\text{mod}\, p)
\]
%
\end{theorem}
%
\begin{itemize}
\item cannot be used as a primality test
\item $\Leftarrow$ does not hold
\end{itemize}
%
}


%%%%%%%%%%%%%%%%%%%%%%%%%%%%%%%%%%%%%%%%%%%%%%%%%%%%%%%%%%%%%%%%%%%%%%

\frame{
%\frametitle{Pollard (p-1) factorization method}
%
\begin{itemize}
\item invented by John Pollard in 1974
\item is a derivative of the Pollard rho method
\item suitable for n that is $B$-smooth for $p-1$
\begin{itemize}
\item A smooth (or friable) number is an integer which factors completely into small prime numbers. For example, a 7-smooth number is a number whose prime factors are all at most 7.
\end{itemize}
\item uses that for any $a$ and $\forall p \in \mathbb{P} : a^{p-1} \,\equiv \, 1\:(\text{mod}\, n)$, i.e. $a^{p-1} -1 \,\equiv \, 0\:(\text{mod}\, n)$
\end{itemize}
%
algorithm of Pollard�s rho method:
%
\begin{enumerate}
\item pick two random numbers: $x \:(\text{mod}\, n) \text{ and } y \:(\text{mod}\, n)$
\item If $x-y = 0 \:(\text{mod}\, n)$ we found a factor $gcd(x-y, n)$, else go to step 1
\end{enumerate}
%
}
%
%%%%%%%%%%%%%%%%%%%%%%%%%%%%%%%%%%%%%%%%%%%%%%%%%%%%%%%%%%%%%%%%%%%%%%
%
\frame{
%\frametitle{Pollard (p-1) factorization method}
%
\textbf{Algorithm of the Pollard (p-1) factorization algorithm}
%
\begin{enumerate}
\item Input: $n \ge 2$
\item choose $a$ with $1 \leq a \leq n-1$ and arbitrary $B \in \mathbb{N}$
\item $\forall q \in \mathbb{P}, q \leq B$:
\begin{itemize}
\item $a := a^q \:(\text{mod}\, n)$
\item $p := gcd(a-1, n)$
\item if  $p \mid n$ break
\item else select new $a$ and go to step 3
\end{itemize}
\item return $p$
\end{enumerate}
%
}

%%%%%%%%%%%%%%%%%%%%%%%%%%%%%%%%%%%%%%%%%%%%%%%%%%%%%%%%%%%%%%%%%%%%%%

%\subsection{general number filed sieve}
%
%\frame{
%\frametitle{general number field sieve}
%
%\begin{itemize}
%\item most efficient known algorithm aside from Shor�s algorithm
%\item is a generalization of special number field sieve (NFS)
%\item special NFS can only factor certain numbers
%\item running time is super-polynomial and subexponenetial
%\item like the quadratic sieve it also is based on an enhancement of Fermat�s difference of squares technique (introduced by Maurice Kraitchik in the 1920s)
%\begin{itemize}
%\item instead of finding $x$ and $y$ such that $x^2-y^2 = n$ it suffices to find $x$ and $y$ such that $x^2 \, \equiv \, y^2\:(\text{mod}\, n) $
%\end{itemize}
%\item algorithms with subexponential running time are very complex and will not be presented in detail in the paper
%\end{itemize}
%}
%
%%%%%%%%%%%%%%%%%%%%%%%%%%%%%%%%%%%%%%%%%%%%%%%%%%%%%%%%%%%%%%%%%%%%%%

%%%%%%%%%%%%%%%%%%%%%%%%%%%%%%%%%%%%%%%%%%%%%%%%%%%%%%%%%%%%%%%%%%%%%%

\section{RSA-challenge}
%
\begin{frame}[fragile]
\frametitle{RSA-challenge}
%
\begin{itemize}
\item in 2009 RSA-768 was factored over the span of two years
\item it is 768 bits and 232 decimal digits of size
\item it is the largest solved RSA-challenge so far
\item calculated with the general number field sieve
\end{itemize}
%

RSA-768 = 334780716989568987860441698482126908177047949837\\
137685689124313889828837938780022876147116525317\\
43087737814467999489\\
x\\
367460436667995904282446337996279526322791581643\\
430876426760322838157396665112792333734171433968\\
10270092798736308917


\end{frame}
%
%%%%%%%%%%%%%%%%%%%%%%%%%%%%%%%%%%%%%%%%%%%%%%%%%%%%%%%%%%%%%%%%%%%%%%



%%%%%%%%%%%%%%%%%%%%%%%%%%%%%%%%%%%%%%%%%%%%%%%%%%%%%%%%%%%%%%%%%%%%%%%%%%%%%%%%%%%%%%%%%%%%%%%%%%%%%%%%%%%%%%%%%%%%%%%%
\section{Further Information}
%%%%%%%%%%%%%%%%%%%%%%%%%%%%%%%%%%%%%%%%%%%%%%%%%%%%%%%%%%%%%%%%%%%%%%%%%%%%%%%%%%%%%%%%%%%%%%%%%%%%%%%%%%%%%%%%%%%%%%%%

\frame{
\frametitle{For Further Reading}
\begin{thebibliography}{aaaaaaaaaaaaaaaaa}
\beamertemplatearticlebibitems
\bibitem{Rempe:2009}
Lasse Rempe, Rebecca Waldecker
\newblock{\em Primzahltests f�r Einsteiger}
\newblock 1. Auflage, Vieweg+Teubner Verlag, Wiesbaden 

\beamertemplatearticlebibitems
\bibitem{Dietzfelbinger:2004}
Martin Dietzfelbinger
\newblock{\em Primality Testing in Polynomial Time - From Randomized Algorithms to "`PRIMES is in P"'}
\newblock 1.Auflage, Springer Verlag, Berlin Heidelberg  

\beamertemplatearticlebibitems
\bibitem{Brent}
Richard P. Brent
\newblock{\em Recent Progress and Prospects for Integer Factorisation Algorithms}

\end{thebibliography}
}

%%%%%%%%%%%%%%%%%%%%%%%%%%%%%%%%%%%%%%%%%%%%%%%%%%%%%%%%%%%%%%%%%%%%%%%%%%%%%%%%%%%%%%%%%%%%%%%%%%%%%%%%%%%%%%%%%%%%%%%%
\frame{
\frametitle{For Further Reading}
\begin{thebibliography}{aaaaaaaaaaaaaaaaa}
\beamertemplatearticlebibitems
\bibitem{Kleinjung:2010}
Thorsten Kleinjung and Kazumaro Aoki and Jens Franke and Arjen Lenstra and Emmanuel Thom� and Joppe Bos and Pierrick Gaudry and Alexander Kruppa and Peter Montgomery and Dag Arne Osvik and Herman te Riele and Andrey Timofeev and Paul Zimmermann
\newblock{\em Factorization of a 768-bit RSA modulus}
\newblock Cryptology ePrint Archive, Report 2010/006
\newblock available on \href{https://eprint.iacr.org/2010/006}, 2018

\end{thebibliography}
}

% Define the end of the document:
\end{document}
